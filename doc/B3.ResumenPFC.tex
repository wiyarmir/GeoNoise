%%%%%%%%%%%%%%%%%%%%%%%%%%%%%%%%%%
% Página de resumen del proyecto %
%%%%%%%%%%%%%%%%%%%%%%%%%%%%%%%%%%

\thispagestyle{empty}
\begin{center}
	\Large \sffamily
	Universidad de Málaga\\
	Escuela Técnica Superior de Ingeniería de\\
	Telecomunicación
\end{center}

\bigskip

\begin{center}
	\Huge\scshape
	\pfctitlename
\end{center}

\bigskip

\begin{center}
	\textbf{REALIZADO POR}\\
	\textsf{\pfcauthorname}
\end{center}

\medskip

\begin{center}
	\textbf{DIRIGIDO POR}\\
	\textsf{\pfctutorname}
\end{center}

\vfill

\begin{minipage}{\textwidth}
\textbf{Dpto. de:} Ingeniería de Comunicaciones (IC)

\medskip

\textbf{Palabras clave:} ruido, android, aplicación, geolocalización, mapa, calor

\medskip

\textbf{Titulación:} Ingeniería Técnica de Telecomunicación Especialidad en Sonido e Imagen

\medskip

\textbf{Resumen:}
	El presente proyecto consiste en desarrollar una aplicación móvil para el sistema operativo Android, cuya funcionalidad es la medición del nivel de ruido asociado a un punto geográfico, y la representación del mismo superpuesto a un mapa geográfico

\begin{center} Málaga, \today\end{center}
\end{minipage}

\blankpage