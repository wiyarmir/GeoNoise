%%%%%%%%%%%%%%%%%%%%%%%%%%%%%%%%%%%%%%%%%%%%%%%%%%%%%%%%%%%%%%%%%%%
%%% Documento LaTeX 																						%%%
%%%%%%%%%%%%%%%%%%%%%%%%%%%%%%%%%%%%%%%%%%%%%%%%%%%%%%%%%%%%%%%%%%%
% Título:		Introducción
% Autor:  	Ignacio Moreno Doblas
% Fecha:  	2014-02-01
% Versión:	0.5.0
%%%%%%%%%%%%%%%%%%%%%%%%%%%%%%%%%%%%%%%%%%%%%%%%%%%%%%%%%%%%%%%%%%%

%\chapterbeginx{Introducción, Objetivo y Visión general}
\chapterbeginx{Introducción y visión general}
\minitoc

\begin{sinopsis}
\label{sec:intro:sinop}
	Éste es el capítulo de introducción, donde se explica todo lo que un lector externo necesita para entender el resto de la documentación.
	
	El objetivo explica lo que persigue el proyecto, su finalidad.\nli
	El \miindex{estado del arte} explica la situación actual del entorno en el que este proyecto\footnote{En adelante, se utiliza la palabra \tit{proyecto} como sinónimo de TFG/TFM/PFC, según se aplique (Nota del autor).} se desenvuelve.\nli
	Las metodologías y directrices seguidas se centran en qué procedimientos se han utilizado durante el desarrollo del proyecto.\nli
	La estructura del documento describe los capítulos de los que se compone, incluyendo apéndices e información adicional.\nli
	Por último, el ámbito de aplicación completa el entorno de utilización del proyecto.
	
	Aunque estos cinco apartados no son obligatorios, al menos es recomendable considerar estos conceptos en el capítulo de introducción como una guía básica.
	
	Tampoco es obligatorio usar el entorno \LaTeX\ \ttw{minitoc} para cada capítulo.\nli
	En caso de no querer usarlo, tan sólo hay que comentar la línea \ttw{\textbackslash \miindex{minitoc}}.
	
	Igualmente, esta sección inicial de Sinopsis no es obligatoria, se puede suprimir si no se desea.
	
	Por extensión y en general, esta plantilla es una guía cuyo objetivo es facilitar la realización del proyecto, no un reglamento estricto ni rígido.
\end{sinopsis}
 
\sectionx{Resumen y objetivos del Proyecto}
\label{sec:intro:obj}
Cosas

\sectionx{Motivación}
    Es difícil encontrar actividades que no generen cierto nivel sonoro, ya sean naturales o producidas por el ser humano. Estos sonidos pueden ser categorizados en base a muchos parámetros, como lugar, duración, tipo, etc. pero cuando el sonido es molesto y no deseado, estamos hablando de ruido. Este ruido puede llegar a ser perjudicial para la audición, el físico y el psique de seres vivos, convirtiéndose en contaminación acústica.
    
    La contaminación acústica es hoy día un factor clave en el deterioro de la calidad ambiental de un territorio. Según estudios de la Unión Europea (2005) “80 millones de personas están expuestas diariamente a niveles de ruido ambiental superiores a 65dBa y otros 170 millones, lo están a niveles entre 55-65dBa”
    
    Normalmente, las mediciones de ruido conllevan el uso de múltiples y muy caros aparatos tales como sonómetro, calibrador, pantalla anti-viento, trípode, micrófono, preamplificador, etc. pero se puede tomar provecho de la alta penetración en el mercado que tienen los teléfonos inteligentes y conseguir, a costa de pérdida de precisión, una mayor facilidad y simplicidad a la hora de realizar mediciones.
    
    La vasta mayoría de estos dispositivos incorporan de serie un receptor GPS, micrófono y almacenamiento, además de acceso a internet y pantalla táctil.\

\sectionx{Estado del arte}
	Un proyecto se realiza sobre un \miindex{estado de la técnica} que debe explicarse para entender mejor conceptos tales como los problemas existentes o cuáles son las soluciones que se emplean hasta la fecha actual.

	El \miindex{estado del arte}, a veces llamado estado de la técnica, suele estar presente en este tipo de documentos.
  
\sectionx{Metodología y directrices seguidas}
	Durante la elaboración del proyecto, se siguen procedimientos que el lector necesita conocer para entender de forma integral todo el documento.

\sectionx{Estructura del documento}

\textbf{Introducción teórica}

En el capítulo 2 se hará una introducción a los conceptos teóricos requeridos para la comprensión del trabajo realizado.
Se explicarán los parámetros acústicos tratados a lo largo de la memoria.
A su vez, también se explicarán conceptos relativos a la programación en Android, tratados a lo largo del proyecto.

\textbf{Implementación}

En el capítulo 3 se explicará cómo se ha llevado a cabo toda la implementación de la aplicación, empezando por un análisis de los requisitos necesarios para la realización de esta, y siguiendo con los aspectos relevantes a la programación de la aplicaciónÑ métodos, interfaz gráfica, algoritmos, etc.

\textbf{Plan de pruebas y verificación}

Aquí se explicarán los procedimientos llevados a cabo para cerciorar el correcto funcionamiento de la aplicación.

\textbf{Conclusiones y trabajo futuro}

En este apartado se pretende representar las conclusiones obtenidas durante la realización de este proyecto así como plasmar posibles ideas que se consideran interesantes de cara a una futura continuidad en el desarrollo de la aplicación.

\sectionx{Ámbito de aplicación}
	Por último, completando los apartados anteriores, se explican las áreas de las que se compone el proyecto.

\chapterend